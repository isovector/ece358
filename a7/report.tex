% Due

\documentclass[a4paper]{article}

\usepackage[english]{babel}
\usepackage[utf8]{inputenc}
\usepackage{amsmath}
\usepackage{graphicx}
\usepackage{parskip}
\usepackage{amssymb}
\usepackage{mathtools}
\usepackage{upgreek}
\usepackage{svg}
\usepackage{todonotes}

\DeclarePairedDelimiter\ceil{\lceil}{\rceil}
\DeclarePairedDelimiter\floor{\lfloor}{\rfloor}

\title{CS454 Assignment 4}
\author{Ariel Weingarten --- 20366892}
\date{\today}

\begin{document}

\maketitle

%(10 points) Let G be a weighted undirected graph that represents a network.
% (a) Give an example of G and events as discussed below, that demonstrate the following behaviour. Assume that poisoned reverse is in effect. Initially, running a Distance-Vector routing protocol on G results in no routing loops. After we converge, events occur that cause the routing protocol to be run again among some nodes of G, and we end up with a routing loop that comprises 4 nodes. (i.e., for some destination, A forwards to B, which forwards to C, which forwards to D, which in turn forwards to A.)
% (b) In what way would using path vector routing address the above behaviour demonstrated by G? (In path vector routing, not only distance vectors are sent between routers, but also the paths that correspond to those distances.)
(b) In what way would using path vector routing address the above behaviour demonstrated by G? (In path vector routing, not only distance vectors are sent between routers, but also the paths that correspond to those distances.)
\section{Question 1}

% (5 points) Let H: {0,1}∗ −→ {0,1}c, for some constant c, be a uniform random function. What is the number of evaluations of H(·) we expect to have to perform to find three inputs that collide?
\section{Question 2}

% (5 points) Let the encryption scheme for a public-private key-pair <k, k−1> be: E(x) = {r}k || G(r) ⊕ x
% where r is a random number that is generated anew for each instance of encryption, G(·) is a random number generator that is a function of the input seed, and “||” is string-concatenation. Show that the scheme is not secure against a chosen ciphertext attack mounted by a polynomial-time adversary. (Hint: one way to fix the problem is to change the encryption operation to E(x) = {r}k || G(r) ⊕ x || H(r||x), where H is a hash function.)
\section{Question 3}

% (5 points) Consider the zero-knowledge proof for quadratic-residuosity from my notes, which, in turn, is from Goldwasser et al. [GMR89]. It appears that the server should always choose b = 1 because that is the only option that forces Alice to respond with something that has to do with y. Show that if Alice knows that the server will always choose b = 1, then she can successfully participate in the protocol for any n even when she does not know w.
\section{Question 4}

%(5 points) From Section 4, Example 3.4 of Abadi and Needham [AN95]: “We leave the construction of an attack as an exercise for the reader.” Construct an attack. (You can use their proposed resolution as a hint.)
\section{Question 5}

% (5 points) In Section 5, Example 5.2 of Abadi and Needham [AN95], fix the message so that B can be convinced that A knows X. Your solution should have all the security properties of the existing solution plus this.
\section{Question 6}

% (5 points) In Section 6.2, Example 9.1 of Abadi and Needham [AN95], suppose that Message 5 is A −→
% B:{Nb}Kab (i.e.,withoutthe“+1”).Showanattackontheprotocol.
\section{Question 7}

% (5 points) Suppose that a malicious principal E acquires Kab from a prior run of the Needham-Schroeder symmetric-key authentication protocol in Section 6.2, Example 9.1 of Abadi and Needham [AN95]. Show how E can compromise (or “deceive,” as the paper puts it) B as a consequence.
\section{Question 8}


\end{document}
